\section{Results}\label{sec:results}

% ==================================================================
%
% EB writes here
%
\begin{figure}[H]
    \centering
    \includegraphics[width=0.9\linewidth]{examples/tests_eb/figs/cumsum_pca.pdf}
    \caption{A plot of the cumulative variance for each principle component included in our new, dimensionality reduced features. We concluded that 70 features was sufficient to capture just above 85 percent of the variance in our data.}
    \label{fig_cumsumpca}
\end{figure}

\begin{figure}[H]
    \centering
    \includegraphics[width=1.1\linewidth]{examples/tests_eb/figs/pca0_pca1.pdf}
    \caption{The two first principal components out of 70 plotted against each other. Already here we can see some weak signs of clusters}
    \label{fig:pca0pca1}
\end{figure}

\begin{figure}[H]
    \centering
    \includegraphics[width=1.1\linewidth]{examples/tests_eb/figs/umap.pdf}
    \caption{A UMAP plot to explore non-linear relations in our data (TODO - read up on UMAP). Here we can clearly see how our extracted features cluster together, yet we still do not have 14 distinct clusters.}
    \label{fig:enter-label}
\end{figure}

% ==================================================================